\documentclass[draft]{article}

% if you need to pass options to natbib, use, e.g.:
% \PassOptionsToPackage{numbers, compress}{natbib}
% before loading nips_2016
%
% to avoid loading the natbib package, add option nonatbib:
% \usepackage[nonatbib]{nips_2016}

\usepackage{nips_2016}

\usepackage{amsmath} % used for matrices in math mode

% to compile a camera-ready version, add the [final] option, e.g.:
% \usepackage[final]{nips_2016}

\usepackage[utf8]{inputenc} % allow utf-8 input
\usepackage[T1]{fontenc}    % use 8-bit T1 fonts
\usepackage{hyperref}       % hyperlinks
\usepackage{url}            % simple URL typesetting
\usepackage{booktabs}       % professional-quality tables
\usepackage{amsfonts}       % blackboard math symbols
\usepackage{nicefrac}       % compact symbols for 1/2, etc.
\usepackage{microtype}      % microtypography

\usepackage{graphicx}  % Required for including images


\title{Pre-Training CNN's Using Convolutional Autoencoders}

% The \author macro works with any number of authors. There are two
% commands used to separate the names and addresses of multiple
% authors: \And and \AND.
%
% Using \And between authors leaves it to LaTeX to determine where to
% break the lines. Using \AND forces a line break at that point. So,
% if LaTeX puts 3 of 4 authors names on the first line, and the last
% on the second line, try using \AND instead of \And before the third
% author name.

\author{
  David S.~Hippocampus\thanks{Use footnote for providing further
    information about author (webpage, alternative
    address)---\emph{not} for acknowledging funding agencies.} \\
  Department of Computer Science\\
  Cranberry-Lemon University\\
  Pittsburgh, PA 15213 \\
  \texttt{hippo@cs.cranberry-lemon.edu} \\
  %% examples of more authors
  %% \And
  %% Coauthor \\
  %% Affiliation \\
  %% Address \\
  %% \texttt{email} \\
  %% \AND
  %% Coauthor \\
  %% Affiliation \\
  %% Address \\
  %% \texttt{email} \\
  %% \And
  %% Coauthor \\
  %% Affiliation \\
  %% Address \\
  %% \texttt{email} \\
  %% \And
  %% Coauthor \\
  %% Affiliation \\
  %% Address \\
  %% \texttt{email} \\
}

\begin{document}
% \nipsfinalcopy is no longer used

\maketitle

\begin{abstract}
  We are comparing the effect of pretraining.
\end{abstract}

\section{Introduction}
\section{Related Work}
\section{Setup \& Context}
  
  In this paper, we evaluate the influence of CNN pre-training using convolutional autoencoders on the network's out-of-sample accuracy. To achieve this, we train, in a first step, a convolutional autoencoder on a chosen dataset and then, in a second step, use it's convolution layer weights to initialize the convolution layers of a CNN. After training, we compare the CNN's test set accuracy to a reference network that was trained under the same conditions with randomly initialized convolution weights. 

  \begin{figure}[h]
    \centering
    \includegraphics[width=0.5\linewidth]{../graphics/setup.png}
    \caption{Experimental Setup}
    \label{fig:experimental_setup}
  \end{figure}

  A visualization of the weight transfer can be seen on the schematic in figure ~\ref{fig:experimental_setup}. 

  \begin{itemize}
    \item CAE on full / CNN on part of the data (expensive labels scenario)
  \end{itemize}

  \subsection{Autoencoder Architecture + Training}

    \paragraph{Architecture}
    The convolutional autoencoder first uses several convolution layers and pooling layers to transform the input to a feature map representation and then reconstructs the input using strided transposed convolutions (TODO: similar to Zeiler / first GAN layers? + citation). In our experiments, the encoding part consisted of:

    in general: conv layers have (1,1,1,1) strides and we are performing a full convolution (out hei/wid = in hei(wid))

    input (batchsize, height, width, numchannels) -> conv1 (5x5 filter) (batchsize, height, width, 100) -> activation function (scaled tanh) (b,h,w,100) -> max pooling (2x2, (1,2,2,1) strides) (b, h/2, w/2, 100) -> conv2 (5x5 filter) (b, h/2, w/2, 150) -> activation function (Scaled tanh) (b, h/2, w/2, 150) -> max  pooling (2x2 filter, (1,2,2,1) strides (b, h/4, w/4, 150) -> conv3 (3x3 filter), (b, h/4, w/4, 200) -> activation function (scaled tanh) (b, h/4, w/4, 200) 

    \begin{itemize}
      \item Upsampling + Reconstruction
    \end{itemize}

    \paragraph{Activation Function} We tried 3 different activations (sigmoid, scaled tanh and relu). For our main experiments, we decided on the the scaled tanh as used in (TODO: add citation). We will talk about our ReLU experiments in (TODO: add link). With scaled tanh we mean the tanh function rescaled and shifted to the [0,1] output range. This corresponds to the function $$scaledtanh(x) = \frac{1}{2}tanh(x) + \frac{1}{2}$$ which has a generally sigmoidal shape but a stronger gradient around $x = 0$. In our experiments, CAEs seemed faster to train using this function, however we did not conduct any experiments to verify this.

    \paragraph{Regularization} In general, a full convolutional layer could easily learn a simple point filter such as $k = \begin{smallmatrix} 0&0&0\\ 0&1&0 \\ 0&0&0 \end{smallmatrix}$ (for 1d input) that copies the input onto a feature map. While this would later simplify a perfect reconstruction of the input, the CAE did not find any more suitable representation for our data. To prevent this problem, some kind of regularization is needed, several methods can be found in the literature, most of them involving some kind of sparsity constraint for the representation layer (TODO: reference sparse CAE paper and more regularization ideas?). In (TODO: cite stacked conv auto), the authors use max-pooling as an elegant way to enforce the learning of plausible filters without the need for any further regularization. Since the CNN architecture we are going to use later will contain pooling layers anyways, we sticked with this technique and chose not to add any noise to our inputs either since this seemed not to help the emergence of more natural filters. (TODO: describe what is a natural filter for us). 

    \paragraph{Training} While (TODO: add citations) used a layer-wise training for convolutional autoencoders, we achieved good results by training the whole auto-encoder at once. (TODO: maybe this made it more difficult / longer to train?). For the \emph{MNIST} and \emph{CIFAR-10} datasets, the autoencoders were trained with simple gradient descent using as learning rate of $lr = 0.5$ (TODO: VERIFY THAT!), for the CKPLUS dataset, we used the adagrad optimizer with the same initial learning rate $lr$. (TODO:VERIFY THAT!)

    \paragraph{Loss Function} MSE, cross-entropy did not give us good results

    \paragraph{Weight initialization} The convolutional filters were initialized using a truncated normal distribution with mean $\mu = 0.001$ and standard deviation $\sigma = 0.05$, all bias values were initialized with the constant $0.001$

  \subsection{Pre-Training and Test Methodology}

    \paragraph{CNN architecture} The CNN uses the same architecture as the CAE encoding for its convolution layers (feature extraction) and then uses a fully-connected layer with size 384 followed by an activation function (scaled tanh), a second fully-connected layer of size 10 and a softmax layer for classification. 

    \paragraph{Weight Initialization}

    The convolutional layer weights (filters) and biases of the pre-trained networks were initialized with the values obtained from the convolutional autoencoders.

    The reference CNN's convolution weights were initialized with mean $\mu = 0 $ and standard deviation $\sigma = 0.2$, the biases with the constant value $b =  0.0001$.

    The fully-connected layers were initialized randomly for all networks, the weights with a truncated normal distribution with mean $\mu = 0$ and standard deviation $\sigma = 0.001$ and the biases with the constant value $0.0001$.
    This means that for all experiments, some parts of the networks are randomly initialized, all weights for the reference networks and the fully-connected weights for the pre-trained networks. 

    \paragraph{CNN Training}
    (TODO: optimizer, step size, convergence criterion, amount of epochs per dataset)

    For each datset, we split the available training data into training and evaluation set and used the evluation set to track the training progress and adjust hyperparameters. We kept track of both cross-entropy error and accuracy on the evaluation set and stopped training when these values converged. Aftewrards, we evaluated the networks accuracy once on the held-out test set and used the obtained value as a result. 

    For the larger datasets \emph{MNIST} and \emph{CIFAR-10}, we conducted several experiments restricting the amount of available training data inspired by (TODO: cite stacked conv autoencoders). The autoencoder used for pre-training was always trained on the whole dataset. 

    %\begin{itemize}
      %\item Network Architectures
      %\item Weight Transfer
      %\item weight initialization (all layers)
      %\item CNN Training
      %\item Test Methodology (Train + Validation + Test Sets)
      %\item 1k/10k/full splits but always full dataset for CAE
    %\end{itemize}

  \subsection{Evaluation}
    For each dataset and training size, we trained (TODO adjust to real number) k pre-trained CNNs and k randomly initialized reference CNNs. (TODO: add significance test?)
  

\section{Experiments}

  \subsection{Datset Descriptions and Challenges}

    \begin{itemize}
    \item{MNIST}
        MNIST is the standard dataset used for image classification. It consists of scanned postal code numbers (?),
        which are 28 by 28 pixel grayscale images.
    \item{CIFAR-10}
        The CIFAR dataset is made out of natural scenes, which are 32x32 colorized images. The CIFAR-10 variant,
        which we used, is built as a classification dataset with 10 specified labels (cars, cats...)
    \item{CK+}
        The Extended Cohn-Kanade (CK+) dataset was the most interesting dataset during our project. Unlike the other
        two datasets, no pre-training experiments have been done on CK+, as far as we know.
        As the CK+ is not readily prepared for running those kinds of experiments, we had to put some work into this ourselves.
        We decided to not take the full images, but to use the provided facial landmarks, to put the faces
        into a bounding box of size (...).
         
    \end{itemize}

  \subsection{Autoencoder Training}
  For the autoencoder training we took each of the three datasets' images, reshaped them into a vector and used
  these values as input and target output for a convolutional autoencoder.
  \subsection{Pretraining and Classification}
  The general idea of pretraining a classification neural network, is to transfer the encoding weights of
  the CAE to the CNN. So we took the convolutional filter weights, which we generated during the autoencoder training,
  and used them as initial values for the convolutional layers for the classification network.

  As the MNIST and CIFAR-10 datasets consist of a lot more images than CK+, we decided to follow the approach of the paper
  and take a 1k and 10k subset for these two. So while we still took all training images for building a decent autoencoder,
  the CNN could only use some of the images. This is especially interesting to check the hypothesis, that pre-training makes
  more sense, when you have less training images.

  To validate our pretraining experiments, in each run we set up a CNN with randomly initialized filters and a CNN which
  uses the filters trained by the CAE. Both networks have randomly initialized weights for their dense layers.


\section{Discussion and Conclusion}
  
  \begin{itemize}

    \item Evaluation of the Results         (Does it work)
    \item Evaluation of further experiments (When does it work, comparison with relu etc)

  \end{itemize}

\end{document}
